%%% LaTeX Template: Two column article
%%%
%%% Source: http://www.howtotex.com/
%%% Feel free to distribute this template, but please keep to referal to http://www.howtotex.com/ here.
%%% Date: February 2011

%%% Preamble
\documentclass[	DIV=calc, paper=letter, fontsize=11pt,twocolumn]{scrartcl} % KOMA-article class

\usepackage{lipsum}													% Package to create dummy text

\usepackage[english]{babel}											% English language/hyphenation
\usepackage[protrusion=true,expansion=true]{microtype}				% Better typography
\usepackage{amsmath,amsfonts,amsthm}							% Math packages
\usepackage[pdftex]{graphicx}									% Enable pdflatex
\usepackage[svgnames]{xcolor}									% Enabling colors by their 'svgnames'
\usepackage[hang, small,labelfont=bf,up,textfont=it,up]{caption}	% Custom captions under/above floats
\usepackage{epstopdf}												% Converts .eps to .pdf
\usepackage{subfig}													% Subfigures
\usepackage{booktabs}												% Nicer tables
\usepackage{fix-cm}													% Custom fontsizes
\usepackage{pifont}													% zapf dingbats


%%% Custom sectioning (sectsty package)
\usepackage{sectsty}											% Custom sectioning (see below)
\allsectionsfont{%												% Change font of al section commands
	\usefont{OT1}{phv}{b}{n}%									% bch-b-n: CharterBT-Bold font
	}

\sectionfont{%													% Change font of \section command
	\usefont{OT1}{phv}{b}{n}%									% bch-b-n: CharterBT-Bold font
	}



%%% Headers and footers
\usepackage{fancyhdr}											% Needed to define custom headers/footers
	\pagestyle{fancy}											% Enabling the custom headers/footers
\usepackage{lastpage}	

% Header (empty)
\lhead{}
\chead{}
\rhead{}
% Footer (you may change this to your own needs)
\lfoot{\footnotesize \texttt{rui-han.com} \textbullet Research Reading Summary}
\cfoot{}
\rfoot{\footnotesize page \thepage\ of \pageref{LastPage}}	% "Page 1 of 2"
\renewcommand{\headrulewidth}{0.0pt}
\renewcommand{\footrulewidth}{0.4pt}

%%% Creating an initial of the very first character of the content
\usepackage{lettrine}
\newcommand{\initial}[1]{%
     \lettrine[lines=3,lhang=0.2,nindent=0em]{
     				\color{DarkGoldenrod}
     				{\textsf{#1}}}{}}

%%% Title, author and date metadata
\usepackage{titling}									% For custom titles
\newcommand{\HorRule}{\color{DarkGoldenrod}%			% Creating a horizontal rule
									  	\rule{\linewidth}{1pt}%
										}

\pretitle{\vspace{-30pt} \begin{flushleft} \HorRule 
				\fontsize{50}{50} \usefont{OT1}{phv}{b}{n} \color{DarkRed} \selectfont 
				}
\title{Selective Symbolic Execution}	% Title of your article goes here
\posttitle{\par\end{flushleft}\vskip 0.5em}

\preauthor{\begin{flushleft}
					\large \lineskip 0.5em \usefont{OT1}{phv}{b}{sl} \color{DarkRed}}
\author{Vitaly Chipounov, Vlad Georgescu, etc. } % Author name goes here
\postauthor{\footnotesize \usefont{OT1}{phv}{m}{sl} \color{Black} 
					EPFL % Institution of author
					\par\end{flushleft} Rate: \ding{72}\ding{72}\ding{72}\ding{72}\ding{73} \HorRule}

\date{\today}											

%%% Begin document
\begin{document}
\maketitle
\thispagestyle{fancy} 			% Enabling the custom headers/footers for the first page 
% The first character should be within \initial{}
\initial{T}\textbf{his paper discuss the S2E method very clear in a relative higher
level. In the paper, it discuss the symbolic execution as following: the
symbolic engine will represent the program’s memory in an engine-specific data
structure and, when the execution reaches a branch whose condition involves
symbolic values, the engine forks the program states, such as each path has
its own private program state.}

\section*{Challenges}

One of the challenge with SE is that it has to deal with the program’s
interaction with the environment. This challenge also prevent SE from being
efficiently executed.

S2E is a virtual execution platform that allow user to specify a scope of
interest within a system’s execution space, and then focuses CPU and memory
resources on symbolically executing that scope only. This is a great idea to
make SE to run in real system. By doing so, S2E pruning parts of the execution
tree that the developer would not even look at once execution completed.

\section*{Use cases}

The use case for this is broad, including (1) Testing in complex environment,
(2) Fine-grain module testing, (3) Data-driven testing, (4) Hybrid-input
testing, (5) Reproducing user-reported bugs, (6) Dynamic failure analysis, (7)
Failure avoidance, (8) Reverse engineering programs.

For (1), Think about symbolic execute Firefox, When the input to FF is
symbolic, FF also call system libs with symbolic, how to ensure that the
return value to FF is vilad? This means that the symbolic to concrete
conversion from FF to environment is one way.

For (2), Think about doing unit testing for one of FF’s properties.  For (3),
Think about during code development, you want to change a data structure that
has been extensively touched throughout the code, how could you ensure the
modified data structure is works fine after the modification.  For (4), This
is similar to the (1), which is to ensure quickly reach some conner case.  For
(5), The metaphor in the paper is great, which says: The details of the bug
report define an envelope of executions, and S2E searches for the culprit path
only within the envelope.  For (6) and (7), they are involving some special
scenario, refer to the papepr 

\end{document}
