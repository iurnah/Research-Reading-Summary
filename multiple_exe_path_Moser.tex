%%% LaTeX Template: Two column article
%%%
%%% Source: http://www.howtotex.com/
%%% Feel free to distribute this template, but please keep to referal to http://www.howtotex.com/ here.
%%% Date: February 2011

%%% Preamble
\documentclass[	DIV=calc, paper=letter, fontsize=11pt,twocolumn]{scrartcl} % KOMA-article class

\usepackage{lipsum}													% Package to create dummy text

\usepackage[english]{babel}											% English language/hyphenation
\usepackage[protrusion=true,expansion=true]{microtype}				% Better typography
\usepackage{amsmath,amsfonts,amsthm}							% Math packages
\usepackage[pdftex]{graphicx}									% Enable pdflatex
\usepackage[svgnames]{xcolor}									% Enabling colors by their 'svgnames'
\usepackage[hang, small,labelfont=bf,up,textfont=it,up]{caption}	% Custom captions under/above floats
\usepackage{epstopdf}												% Converts .eps to .pdf
\usepackage{subfig}													% Subfigures
\usepackage{booktabs}												% Nicer tables
\usepackage{fix-cm}													% Custom fontsizes
\usepackage{pifont}													% zapf dingbats


%%% Custom sectioning (sectsty package)
\usepackage{sectsty}											% Custom sectioning (see below)
\allsectionsfont{%												% Change font of al section commands
	\usefont{OT1}{phv}{b}{n}%									% bch-b-n: CharterBT-Bold font
	}

\sectionfont{%													% Change font of \section command
	\usefont{OT1}{phv}{b}{n}%									% bch-b-n: CharterBT-Bold font
	}



%%% Headers and footers
\usepackage{fancyhdr}											% Needed to define custom headers/footers
	\pagestyle{fancy}											% Enabling the custom headers/footers
\usepackage{lastpage}	

% Header (empty)
\lhead{}
\chead{}
\rhead{}
% Footer (you may change this to your own needs)
\lfoot{\footnotesize \texttt{rui-han.com} \textbullet Research Reading Summary}
\cfoot{}
\rfoot{\footnotesize page \thepage\ of \pageref{LastPage}}	% "Page 1 of 2"
\renewcommand{\headrulewidth}{0.0pt}
\renewcommand{\footrulewidth}{0.4pt}

%%% Creating an initial of the very first character of the content
\usepackage{lettrine}
\newcommand{\initial}[1]{%
     \lettrine[lines=3,lhang=0.2,nindent=0em]{
     				\color{DarkGoldenrod}
     				{\textsf{#1}}}{}}

%%% Title, author and date metadata
\usepackage{titling}									% For custom titles
\newcommand{\HorRule}{\color{DarkGoldenrod}%			% Creating a horizontal rule
									  	\rule{\linewidth}{1pt}%
										}

\pretitle{\vspace{-30pt} \begin{flushleft} \HorRule 
				\fontsize{40}{40} \usefont{OT1}{phv}{b}{n} \color{DarkRed} \selectfont 
				}
\title{Exploring Multiple Execution Paths for Malware Analysis}	% Title of your article goes here

\posttitle{\par\end{flushleft}\vskip 0.5em}

\preauthor{\begin{flushleft}
					\large \lineskip 0.5em \usefont{OT1}{phv}{b}{sl} \color{DarkRed}}
\author{Andreas Moser, Chri Kruegel, Engin Kirda} % Author name goes here

\postauthor{\footnotesize \usefont{OT1}{phv}{m}{sl} \color{Black} 
					Technical University of Vienna % Institution of author
					\par\end{flushleft} Rate: \ding{72}\ding{72}\ding{72}\ding{73}\ding{73} \HorRule}

\date{\today}											

%%% Begin document
\begin{document}
\maketitle
\thispagestyle{fancy} 			% Enabling the custom headers/footers for the first page 
% The first character should be within \initial{}
\initial{I}\textbf{t explore extra path by dynamically tracking the input
data with taint. The system can detect the branch condition with those taint
input (such as current time, file, or the result of a check for internet
connectivity), whenever those taint input is use, it will copy the memory
state and restore these states later. The idea is very similar to the symbolic
execution in klee and s2e. The only difference is this paper use the native
memory manipulation (data structure and page manipulation ) in order to
explore the extra path, which is “fork” in klee and s2e implementation.
Specifically, it maintain three different components: (1) sets of taint input
source, (2) shadow memory, (3) extension to the machine instructions to
propagate the taint label. }

\section*{Their Design}

This paper recorded the system calls invoked by the binary and their
arguments, all this is done in QEMU. The system work in the following ways:
(1) Taint input and dynamically track the input data. (2) Once the input data
is use for control decision, make a snapshot of current memory, (3) Take one
of the path to execute until cover full path, (4) Start another path that
backuped, (similar to depth first search).  Contents of Heading on level 1.

\section*{Specific Design Issues Mentioned}

For (1), To tracking the taint, the system only support linear propagation,
that means when the input related data involving in plus and minus, the label
is propagated linearly. This is not true for multiplication. Also, once a
memory location’s label has been modified, all the other memory location
related to this memory should be updated, This is also the main reason why we
copy the process’s whole memory address in the snapshot. To remember, there is
a constraint system that maintain all the relations between labels.

For(2), The saved snapshot include the complete current virtual memory
contents of the running process. Another more important point is that we also
have to take into account the conditional operation itself, this is help to
pick the next condition to run the code in future. In the paper, it take
labels into a constraint system, which mean not only the input data has
constrain, the label of the input data also has constraint. Otherwise the
alternative branch cannot be explored.

For (3) and (4), it is very straightforward.

\end{document}
